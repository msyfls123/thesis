%!TEX program = xelatex
\documentclass[master,oldfontcfg,euler,twoside,openany]{cugthesis}
% 默认twoside 双面打印
% 将master修改为bachelor, doctor or master (暂时只支持硕士毕业论文模版)
% 要使用adobe字体,添加adobefonts选项
% 使用euler数学字体,如不愿使用,去掉euler
% 使用外文写作,请添加notchinese
% oldfontcfg,使用老板的字体设置,建议初学者开启

% 设置图形文件的搜索路径
\graphicspath{{figures/}}
\newcommand{\tabincell}[2]{\begin{tabular}{@{}#1@{}}#2\end{tabular}}
%仅用于本示例文档中显示特殊字符串
\usepackage{xltxtra}
%仅用于子图
%\usepackage{graphicx}
%\usepackage{subfigure}

%%%%%%%%%%%%%%%%%%%%%%%%%%%%%%
%% 封面部分
%%%%%%%%%%%%%%%%%%%%%%%%%%%%%%

  % 中文封面内容
  \title{MapReduce中间结果重用优化及分析}%一般情况下扉页和封皮、书脊共用一个标题文本,可以不用定义\spinetitle(仅硕博有用), \covertitle(本硕博均有用)和\encovertitle(仅本科有用)。特殊情况见下。
  \spinetitle{MapReduce中间结果重用优化及分析}
  %特殊情况1:本例中\title命令里含有换行控制字符,这会导致制作书脊的时候出现错误,例如如果你注释掉\spinetitle{...}这一行就会报错。这时需要定义一个不含换行等命令的\spinetitle,这并不表示\spinetitle里不能有任何命令——只能使用有限的命令。
  %特殊情况2:本例中标题过长,所以需要缩小书脊标题的字号。
  %特殊情况3:本例中中英文混排,由于tex竖排的原理限制,中英文基线不重合,所以需要人工调整英文的基线。具体调整量根据不同字体有所不同。
  \covertitle{MapReduce中间结果重用优化及分析}
  %\covertitle{中文题目第一行\\中文题目第二行}
  %不要在此调整封皮字体大小! Do not set Cover Page font size here!
  %特殊情况4:本例中\title中含有多个换行,导致标题超过了两行。根据制本厂规定,封皮标题不能超过两行。因此需要定义封皮使用的标题\covertitle. 如果你注释掉这一行,就会发现封皮不符合规定。
  \encovertitle{MapReduce Performance Acceleration and Analytics with Intermediate Results Reusing}
  %\encovertitle{English Title Line 1\\English Title Line 2\\English Title Line 3}
  %不要在此调整封皮字体大小! Do not set Cover Page font size here!
  %特殊情况5:仅本科生有用。本科封皮中有英文标题,不超过三行。与上类似。

  \author{徐\ 锦\ 来}
  \depart{信息工程学院}%系别,硕博请用系代号,本科请用全称如
  %\depart{数理化和信息工程系}
  \major{软件工程专业}%专业,硕博请用全称,本科不需要
  \advisor{罗忠文\ 教授}
  %\coadvisor{姚宏\ 教授}%第二导师,没有请注释掉
  \studentid{120121352}%
  \submitdate{二〇一五年五月}
  %\numsubmitdate{2015.04}
  % 英文封面内容
  \entitle{MapReduce Performance Acceleration and Analytics \\with Intermediate Results Reusing}
  \enauthor{Jinlai Xu}
  \enmajor{Software Engineering}
  \enadvisor{Prof. Zhongwen Luo}
  %\encoadvisor{Prof. Hong Yao}%另外一个导师
  \ensubmitdate{2015.05}
  
%%%%%%%%%%%%%%%%%%%%%%%%%%%%%%%%%%%%%%%%%%%%%%%%%%%%%%%%%%%%%%%%%%%%%
% If you use another language instead of chinese and english, then you
% should define some strings and provide information in your language.
%%%%%%%%%%%%%%%%%%%%%%%%%%%%%%%%%%%%%%%%%%%%%%%%%%%%%%%%%%%%%%%%%%%%%
%  \otherustcstr{zhong guo ke xue ji shu da xue}%A translation of `University of Science and Technology of China' in your language
%  \otherthesisstr{shuo shi xue wei lun wen}%A translation of `A dissertation for doctor(master/bachelor)'s degree' in your language
%  \otherauthorstr{xing ming}%A translation of `Author' in your language
%  \otherdepartmentstr{yuan xi}%A translation of `Department' in your language
%  \otherstudentidstr{xue hao}%A translation of `Student ID' in your language
%  \othersupervisorstr{dao shi}%A translation of `Supervisor' in your language
%  \otherfinishedtimestr{ri qi}%A translation of `Finished Time' in your language
%  \otherspecialitystr{zhuan ye}%A translation of `Speciality' in your language
%  \othertitle{zhong guo ke xue ji shu da xue tong yong xue wen lun wen shi li wen dang}
%  \otherauthor{zhao qian sun}
%  \otheradvisor{zhou wu zheng}
%  \othercoadvisor{feng chen zhu}
%  \othersubmitdate{hou nian ma yue}
%  \othermajor{mou zhuan ye}
%  \otherdepart{mou xi}

\begin{document}

  % 封面
  \maketitle

%特别注意,以下述顺序为准,在对应部分添加文档部件,切勿颠倒顺序:
%本科论文的文档部件顺序是:
%    frontmatter:致谢、目录、中文摘要、英文摘要、
%    mainmatter: 正文章节
%    backmatter: 参考文献或资料注释、附录
%硕博论文的文档部件顺序是:
%    frontmatter:中文摘要、英文摘要、目录、符号说明
%    mainmatter: 正文章节
%    backmatter: 参考文献、附录、致谢、发表论文
%%%%%%%%%%%%%%%%%%%%%%%%%%%%%%
%% 前言部分
%%%%%%%%%%%%%%%%%%%%%%%%%%%%%%
\frontmatter
%\pagenumbering{}
\makeatletter
\ifustc@bachelor
	%%%%%%%%%%%%%%%%%
	%本科论文修改这里
	%%%%%%%%%%%%%%%%%
	% 致谢
	% !TEX root = ../main.tex
\begin{thanks}

作者研究生就读期间得到了各位老师同学的悉心帮助与指导,谢谢你们让我成为一个更好的人!

首先,感谢我的导师周晔老师,周老师从本科起就担任我们专业相关课程的授课老师,课堂上对学子淳淳教诲、言传身教,在科研上精益求精、指导我们向学术领域高峰奋进。在毕业论文开题至写作完成期间,周老师均给予了我莫大的帮助,对论文中存在的问题详细解答,使我受益匪浅。

然后,要感谢本专业其他各位老师。蔡建平老师于本科时指导我参加某全国比赛并收获佳绩,知遇之恩没齿难忘;陈晓鹂老师与我成为了科研学习方面良好的合作伙伴,一起完成了一个又一个科研实践项目;顾湘老师对我严格认真,我于研一时参与顾老师的助教工作感到十分荣幸。还有李理老师、肖畅老师、杨展老师等在我学习期间给予的帮助与支持,谢谢你们!

最后,感谢同寝室的兄弟、专业内其他同学以及我的家人,你们是我生活上最好的伙伴和后盾。

\vskip 18pt

\begin{flushright}

~~~~马申彥~~~~

\today

\end{flushright}

\end{thanks}

	
	%目录部分
	%目录
	\tableofcontents
	%默认表格、插图、算法索引名称分别为“表格索引”、“插图索引”和“算法索引”
	%如果需要自行修改lot,lof,loa的名称,请定义
	%\ustclotname{...}
	%\ustclofname{...}
	%\ustcloaname{...}

	% 表格索引
	\ustclot
	% 插图索引
	\ustclof
	%算法索引 
	%如果需要使用算法环境并列出算法索引,请加入补充宏包。
	\ustcloa
	
	% 摘要
	\begin{abstract}

全景漫游是虚拟现实技术的核心,通过模拟真实场景的全景图或视频以及不同于传统平面媒介的多通道交互手段,用户可以通过全景漫游设备或是手机屏幕等体验到身临其境般的使用体验。通过研究全景漫游相关技术,可以增强交互设计在计算机视觉及工业可视化方向上的应用价值,也帮助提升了包括旅游、通讯、建筑、资源勘探等应用领域的信息可视化程度。

本文从全景漫游发展的现状出发,列举并阐述了现有全景漫游技术的特点及需求,通过比较全景漫游多种应用的操作方式及交互流程发现其有待加强的地方。从人体生理与心理的相关特征特性出发,结合人机工程学和心理学观点深入剖析了全景漫游与人结合的难点和问题,力图以人为设计中心,加强人在全景漫游体验中的参与作用。

从图形化人机界面到信息架构再到功能模块,将全景漫游与真实世界中的游乐场场景进行类比,以交互研究的方式结合相关实例说明信息架构在人机界面的重要作用,论述了信息架构组件化和可视化的必要性。通过建立以漫游功能、导航功能、搜索功能和记忆功能为主的功能模型,列举了部分功能模块的实现要点。

在交互模型方面,以设备交互、导航交互和场景交互为切入点,以信息流的处理过程为例分析了交互操作形式的可反馈性的重要性,提出了用户在交互过程中需要进行信息修正的观点;详细阐述了信息架构的三种组织方式的应用场景,列举了自顶向下、自底向上和不可见的信息架构在模型中的具体体现:上下文感知用以建立导航间的上下文关系、增强信息用以减少用户变换注意的负担、多任务切换的合理形式以在用户记忆中留存相关信息利于语境的切换等。通过建立多层上下文语境增强语境间的联系,促进小语境间切换的适应性以及语境内部信息的流通。以场景中各种漫游路径的实现形式为基础,论述了场景中直观提供场景特殊标识存在的合理性与必要性,并给出了分层级特殊化的指示图标标识。

以实例分析的方式解释并总结了某一例全景漫游系统的交互设计及程序实现的过程,从需求分析到功能架构,以交互模型为指导原则进行可视化设计,主要从导航界面设计、漫游界面设计和功能界面设计三个方向举例交互原型设计,完成漫游系统的设计开发。同时,以 A-Frame 框架为基础,进行了简单的功能程序开发以供设计评价应用。在设计完成后,以收集用户体验数据的形式得到了用户交互操作行为的第一手资料,分析并论证了该交互设计中应用相关设计理论的可行性与成效。

本文提出了以人为中心、可视化交互为主导的全景漫游交互设计思路,从人与全景漫游的关系至全景漫游中信息架构和交互模型的分析论证中分析了全景漫游交互设计中重点功能的实现与细节的处理,为全景漫游相关设计提供了相关分析思路和实践经验。

\textbf{关键词:} 全景漫游、上下文感知、信息流、交互模型

\end{abstract}


\renewcommand{\abstractname}{Abstract}

\begin{abstract}
Panorama roaming is the basement of virtual reality technology. Users can immerse themselves in virtual environment by using panorama roaming device or mobile screen which is based on panorama image or video simulating real environment and multi-channel interaction methods excluding traditional interface media. Through reserach on panorama roaming, the implement value of interaction design in computer graphics and industrial visualization will be enhanced, as well as visualization of information in applyment fields including tourism, communication, architecture and resources exploration.

The thesis firstly lists and demonstrates characters and demands of its technology as current development of panorama roaming technology. The enhanceable details could be found by comparing serveral operative methods and interaction flow of ranorama roaming. The difficulties and problems between panorama roaming and human beings are explained according to ergonomics and psycology, with study on related human physical and psycologic characters. Interaction design will treate human as the middle part of roaming and enhance participation of human in roaming experience.

Methods of interaction researches connected to related applyment practice which are taken in flow of graphic interface, information architecture and functional module, comparison between panorama roaming and amusement park in orderto demonstrate fundemental function 
of information architecture in human-computer interface and its necessity of modularism and visualization of information architecture. The applyment details of specific functional modules are listed through making functional model based on roaming function, navigation function, search function and memory function.

In the part of interaction model, the importance of feedback in interactive operations is analyzed for examplex of information flow process at the point of device interaction, navigation interaction and environment interaction. The view point of user information revision is raised in interaction progress. The applyment fields of three types of interactive operation organization, as known as top-to-bottom, bottom-to-top and invisible forms, are demonstrated with their detailed implement: Context perception carries out context relationship between navigation layers, which enhances information to avoid rapid switching concern of user and keep memories in mind to benefit switch of context by making multi-task switch more reasonable. Multi-layer context is set up to enhance relationship between contexts which benefits both adaption of switch in minor context and context inner information transition. The reasonablity and necessity of existence of direct scene specific mark are demonstrated on base of implement of roaming routes, and layer-specialed guide icon marks are raised for example.

One case of panorama roaming system is detailed and concluded by practice analysis in interaction design and program implement. The work flows from requirement analysis to function architecture, visually developped in guide of interaction model. Interaction prototype design takes examples of navigation interface, roaming interface and function interface design contributing to development of roaming system. A-Frame which known as a web Javascript framework is applied in functional programme development with its extra components to meet the evaluation of design. User experience data is collected immediately after user interaction operation as the original profile which is used for analyzing feasibility and efficiency of related theories in interaction design.


The thesis raised one kind of human-oriented panorama roaming interaction design strategy leading by visualization interaction. Analysis on relationship between human and roaming, demostration of information architecture and interaction model draws the picture of panorama roaming interaction design function implement and detailed process. All above could be fundamental analysis strategy and practical experience for related designs.

\textbf{Keywords: } panorama roaming, context perception, information flow, interaction model

\end{abstract}




%此文件中含有中英文摘要
\else
	%%%%%%%%%%%%%%%%%
	%硕博论文修改这里
	%%%%%%%%%%%%%%%%%
  % 个人简介
  % !TEX root = ../main.tex
\chapter*{作者简介}
 %\thispagestyle{empty}
\begin{spacing}{1.3}

马申彥,男,江苏无锡人,1992 年 4 月出生,
中国地质大学(武汉)机械与电子信息学院在读研究生。本科于中国地质大学(武汉)机械与电子信息学院就读工业设计系。硕士研究生在读期间,致力于产品数字化设计与 Web 技术的研究,曾参与若干校级科研项目,并进行了多次社会实践活动,在产品设计的可视化、设计成果多媒体展示和网页交互设计与程序实现等领域有所研究。

% \input{chapter/pub}
\end{spacing}

\subsection*{参与的科研项目}
\begin{description}
	\item[面向产品设计的 VDP 人机功效模块实验技术研究] 实验数据整理与分析
	\item[基于VDP虚拟仿真软件的机电产品三维可视化设计研究] 学校开放实验室项目,负责人
\end{description}

\vskip 12pt

\subsection*{发表的专利}
\begin{description}
	\item[羽毛球发球器](专利号:ZL.2016305833384)
	\item[折叠式台灯](专利号:ZL.2016305833613)
\end{description}

\vskip 12pt

\subsection*{参与的社会实践}
\begin{description}
	\item[2015.7 - 2015.11] 广东顺德工业设计研究院(设计类实习生)
	\item[2016.3 - 2016.7] 苏州蜗牛数字科技有限公司(前端开发工程师实习)
	\item[2016.10 - 2017.1] 豆瓣阅读(前端开发工程师实习)
\end{description}
  % 摘要
  \begin{abstract}

全景漫游是虚拟现实技术的核心,通过模拟真实场景的全景图或视频以及不同于传统平面媒介的多通道交互手段,用户可以通过全景漫游设备或是手机屏幕等体验到身临其境般的使用体验。通过研究全景漫游相关技术,可以增强交互设计在计算机视觉及工业可视化方向上的应用价值,也帮助提升了包括旅游、通讯、建筑、资源勘探等应用领域的信息可视化程度。

本文从全景漫游发展的现状出发,列举并阐述了现有全景漫游技术的特点及需求,通过比较全景漫游多种应用的操作方式及交互流程发现其有待加强的地方。从人体生理与心理的相关特征特性出发,结合人机工程学和心理学观点深入剖析了全景漫游与人结合的难点和问题,力图以人为设计中心,加强人在全景漫游体验中的参与作用。

从图形化人机界面到信息架构再到功能模块,将全景漫游与真实世界中的游乐场场景进行类比,以交互研究的方式结合相关实例说明信息架构在人机界面的重要作用,论述了信息架构组件化和可视化的必要性。通过建立以漫游功能、导航功能、搜索功能和记忆功能为主的功能模型,列举了部分功能模块的实现要点。

在交互模型方面,以设备交互、导航交互和场景交互为切入点,以信息流的处理过程为例分析了交互操作形式的可反馈性的重要性,提出了用户在交互过程中需要进行信息修正的观点;详细阐述了信息架构的三种组织方式的应用场景,列举了自顶向下、自底向上和不可见的信息架构在模型中的具体体现:上下文感知用以建立导航间的上下文关系、增强信息用以减少用户变换注意的负担、多任务切换的合理形式以在用户记忆中留存相关信息利于语境的切换等。通过建立多层上下文语境增强语境间的联系,促进小语境间切换的适应性以及语境内部信息的流通。以场景中各种漫游路径的实现形式为基础,论述了场景中直观提供场景特殊标识存在的合理性与必要性,并给出了分层级特殊化的指示图标标识。

以实例分析的方式解释并总结了某一例全景漫游系统的交互设计及程序实现的过程,从需求分析到功能架构,以交互模型为指导原则进行可视化设计,主要从导航界面设计、漫游界面设计和功能界面设计三个方向举例交互原型设计,完成漫游系统的设计开发。同时,以 A-Frame 框架为基础,进行了简单的功能程序开发以供设计评价应用。在设计完成后,以收集用户体验数据的形式得到了用户交互操作行为的第一手资料,分析并论证了该交互设计中应用相关设计理论的可行性与成效。

本文提出了以人为中心、可视化交互为主导的全景漫游交互设计思路,从人与全景漫游的关系至全景漫游中信息架构和交互模型的分析论证中分析了全景漫游交互设计中重点功能的实现与细节的处理,为全景漫游相关设计提供了相关分析思路和实践经验。

\textbf{关键词:} 全景漫游、上下文感知、信息流、交互模型

\end{abstract}


\renewcommand{\abstractname}{Abstract}

\begin{abstract}
Panorama roaming is the basement of virtual reality technology. Users can immerse themselves in virtual environment by using panorama roaming device or mobile screen which is based on panorama image or video simulating real environment and multi-channel interaction methods excluding traditional interface media. Through reserach on panorama roaming, the implement value of interaction design in computer graphics and industrial visualization will be enhanced, as well as visualization of information in applyment fields including tourism, communication, architecture and resources exploration.

The thesis firstly lists and demonstrates characters and demands of its technology as current development of panorama roaming technology. The enhanceable details could be found by comparing serveral operative methods and interaction flow of ranorama roaming. The difficulties and problems between panorama roaming and human beings are explained according to ergonomics and psycology, with study on related human physical and psycologic characters. Interaction design will treate human as the middle part of roaming and enhance participation of human in roaming experience.

Methods of interaction researches connected to related applyment practice which are taken in flow of graphic interface, information architecture and functional module, comparison between panorama roaming and amusement park in orderto demonstrate fundemental function 
of information architecture in human-computer interface and its necessity of modularism and visualization of information architecture. The applyment details of specific functional modules are listed through making functional model based on roaming function, navigation function, search function and memory function.

In the part of interaction model, the importance of feedback in interactive operations is analyzed for examplex of information flow process at the point of device interaction, navigation interaction and environment interaction. The view point of user information revision is raised in interaction progress. The applyment fields of three types of interactive operation organization, as known as top-to-bottom, bottom-to-top and invisible forms, are demonstrated with their detailed implement: Context perception carries out context relationship between navigation layers, which enhances information to avoid rapid switching concern of user and keep memories in mind to benefit switch of context by making multi-task switch more reasonable. Multi-layer context is set up to enhance relationship between contexts which benefits both adaption of switch in minor context and context inner information transition. The reasonablity and necessity of existence of direct scene specific mark are demonstrated on base of implement of roaming routes, and layer-specialed guide icon marks are raised for example.

One case of panorama roaming system is detailed and concluded by practice analysis in interaction design and program implement. The work flows from requirement analysis to function architecture, visually developped in guide of interaction model. Interaction prototype design takes examples of navigation interface, roaming interface and function interface design contributing to development of roaming system. A-Frame which known as a web Javascript framework is applied in functional programme development with its extra components to meet the evaluation of design. User experience data is collected immediately after user interaction operation as the original profile which is used for analyzing feasibility and efficiency of related theories in interaction design.


The thesis raised one kind of human-oriented panorama roaming interaction design strategy leading by visualization interaction. Analysis on relationship between human and roaming, demostration of information architecture and interaction model draws the picture of panorama roaming interaction design function implement and detailed process. All above could be fundamental analysis strategy and practical experience for related designs.

\textbf{Keywords: } panorama roaming, context perception, information flow, interaction model

\end{abstract}




%此文件中含有中英文摘要
	% 目录
	\tableofcontents
	%默认表格、插图、算法索引名称分别为“表格索引”、“插图索引”和“算法索引”
	%如果需要自行修改lot,lof,loa的名称,请定义
	%\ustclotname{...}
	%\ustclofname{...}
	%\ustcloaname{...}

	% 表格索引
	\ustclot
	% 插图索引
	\ustclof
	%算法索引 
	%如果需要使用算法环境并列出算法索引,请加入补充宏包。
	%\ustcloa
	
	%符号说明,需要加入补充包
	\include{chapter/denotation}%不是必需的,如果不想列出请注释掉
\fi
\makeatother

%%%%%%%%%%%%%%%%%%%%%%%%%%%%%%
%% 正文部分
%%%%%%%%%%%%%%%%%%%%%%%%%%%%%%
\mainmatter

  \include{chapter/chap-intro}
  \include{chapter/chap-analytics}
  \include{chapter/chap-IntermediateAnalysis}
  \include{chapter/chap-memomr}
  \include{chapter/chap-experiment}
  %\include{chapter/chap-application}
  \include{chapter/chap-summary}
  %自行添加
  %\include{chapter/...}

%%%%%%%%%%%%%%%%%%%%%%%%%%%%%%
%% 附件部分
%%%%%%%%%%%%%%%%%%%%%%%%%%%%%%
\backmatter
  \makeatletter
  \ifustc@bachelor\relax\else
    % 致谢
    % !TEX root = ../main.tex
\begin{thanks}

作者研究生就读期间得到了各位老师同学的悉心帮助与指导,谢谢你们让我成为一个更好的人!

首先,感谢我的导师周晔老师,周老师从本科起就担任我们专业相关课程的授课老师,课堂上对学子淳淳教诲、言传身教,在科研上精益求精、指导我们向学术领域高峰奋进。在毕业论文开题至写作完成期间,周老师均给予了我莫大的帮助,对论文中存在的问题详细解答,使我受益匪浅。

然后,要感谢本专业其他各位老师。蔡建平老师于本科时指导我参加某全国比赛并收获佳绩,知遇之恩没齿难忘;陈晓鹂老师与我成为了科研学习方面良好的合作伙伴,一起完成了一个又一个科研实践项目;顾湘老师对我严格认真,我于研一时参与顾老师的助教工作感到十分荣幸。还有李理老师、肖畅老师、杨展老师等在我学习期间给予的帮助与支持,谢谢你们!

最后,感谢同寝室的兄弟、专业内其他同学以及我的家人,你们是我生活上最好的伙伴和后盾。

\vskip 18pt

\begin{flushright}

~~~~马申彥~~~~

\today

\end{flushright}

\end{thanks}
%硕博致谢部分
    % 发表文章目录
    %\include{chapter/pub}
  \fi
  \makeatother
  % 参考文献
  % 使用 BibTeX
  % 选择参考文献的排版格式。注意ustcbib这个格式不保证完全符合要求,请自行决定是否使用
  \bibliographystyle{cugbib}%{GBT7714-2005NLang-UTF8}
  %\bibliographystyle{gbt7714-2005}%{GBT7714-2005NLang-UTF8}
  \bibliography{bib/ref}
  \nocite{*} % for every item
  % 不使用 BibTeX
  % \include{chapter/bib}

  % 附录,没有请注释掉
  \begin{appendix}
    \include{chapter/chap-req}
  \end{appendix}

  

  

\end{document}
