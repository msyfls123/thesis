\chapter{全景漫游的交互形式}

全景漫游的交互实现有可视化界面、可配戴设备的操作以及语音指令等多种形式,本文所涉及内容主要为可视化交互这种形式,辅以其他交互形式作为参照。可视化图形化人机界面即 GUI 界面由来已久,而最早的计算机则是需要科学家通过输入带孔纸带,计算机通过读取纸带上的小孔获取输入信息来进行运算,现今计算机用户甚至连 70~80 年代常见的命令行界面都未曾见过,如图\ref{fig:gui&cli}。

\begin{figure}[htp]
\centering
\fbox{
\includegraphics[width=.5\textwidth]{gui&cli}
}
\caption{图形化人机界面与命令行界面}
\label{fig:gui&cli}
\end{figure}

从命令行程序到图形化人机界面,用户获得的信息变得更多,但输入信息却减少了很多。无数复杂的程序逻辑隐藏在一次又一次的鼠标点击或是手指点按后面,用户已经不是当年那群使用纸带操作计算机的科学家了。换而言之,用户无法准确记住那么多复杂的计算机命令,更不用提理解计算机内部的程序逻辑,他们只用理解计算机可以提供给他们的功能,例如通过点击某个按钮而无须记住类似“:w”这样的程序命令即可保存文件,并善加利用即可。

而图形化人机界面毕竟信息量相比与命令行界面信息量多出好几个数量级,例如一个全屏 shell 窗口的信息量不会超过 500 字节,但是一个屏幕却有高达 $1920\times1080=2073600$ 个像素,而且图形化人机界面的信息变化效率高,可以用来表现视频动画等信息,其信息复杂度又提高了很多。

综上所述,在图形化人机界面时代随着信息爆炸而造成的信息过载(即信息的处理反馈速度低于信息生产的增长速度
,而造成信息沉积)促使信息界面的设计制作者需从信息架构的角度出发,减少冗余信息,整理信息的分布情况以使用户能更直观高效地获取使用信息。

本章将从信息架构、功能架构两条主线出发,探究全景漫游中可视化交互的可行技术路线,并结合相关理论尝试总结出部分经过实践检验或因技术选择而必然选用的功能模块,以供设计参考。


\section{全景漫游的信息架构}

全景漫游按形式可类比成游乐园,使用者就像游客一样在场景中漫游,由此可以大致描绘出全景漫游的整体信息结构,如图\ref{fig:park}。这种假象的设计可以将使用者置于一种需求和环境相匹配的使用情景中,使用者可以借鉴平时游玩游乐场所积累的记忆,从场景中找寻到熟悉的“场所”以完成体验的过程。这种模式能够尽可能减少乃至消除用户对踏足陌生环境的恐惧感,人的行动预期也可以被规划在场景的设计中。

\begin{figure}[htp]
\centering
\fbox{
\includegraphics[width=.7\textwidth]{park}
}
\caption{全景漫游与游乐场的信息(组织)结构示意}
\label{fig:park}
\end{figure}

\subsection{信息架构形式}
信息架构的形式多样,一般而言可分为:自顶向下、自底向上和不可见的这三种形式。当前数以亿计的网站、应用和系统的信息架构都可以用它们来概括\endnote{罗森菲尔德, 莫尔维莱, 阿朗戈,等. 信息架构:超越Web设计[M]. 电子工业出版社, 2016.}。

自顶向下的信息架构是通常网页所采用的形式,其核心思想是用户已经有了一个需求概念和一个立足点(比如网站的首页),用户从首页出发,通过浏览展现的各种信息选择与自己的需求最为匹配的那个进行浏览(通常是进入下一个页面),然后在下一页面继续重复上述步骤直至找到自己所需的信息。自顶向下的信息架构非常清晰,所有类别目录都被呈现在用户面前,用户只需明确自己的需求就一定能找到“最接近”的那个页面。但缺点是层级嵌套过深时容易使用户丧失进一步浏览的冲动,而往往此时离最后的页面只差一步之遥。自顶而下的信息层级以不超过三级为宜,在全景漫游中,“主界面-类别-场景”的层级刚好是三级,所以大部分导航类的场景(页面)采用这种信息架构是可行的。

自底向上的信息架构主要应用于服务于单个目的的应用中,主要通过建立事物间的联系来帮助使用者在条目间切换。维基百科就是一个很好的例子,每一个词条都有众多相关链接指向其他的百科页面,通过不同的页面联结就构成了完整的知识体系。这种联系的方式也是现代搜索引擎排序的方式,即通过互相引用的计数来判断一个信息的权重,引用计数越高说明这条信息的真实性和准确性越高。全景漫游中单个场景的设计中可应用这种自底向上信息架构,例如在某个著名景点的全景漫游中,在浏览至某壁画前时通过点击其上的热点按钮,可以弹出关于该壁画的相关信息(如图\ref{fig:dunhuang}),这种信息代入方式相比传统的图文模式更容易使人沉浸于情景中,寓教于乐的效果更为理想。

\begin{figure}[htp]
\centering
\fbox{
\includegraphics[width=.7\textwidth]{dunhuang}
}
\caption{全景漫游与景点介绍结合}
\label{fig:dunhuang}
\end{figure}

不可见的信息架构则包含用户输入的部分转换成特定的结构,例如用户的搜索功能等。这种信息架构有时也会影响前两者,比如通过自定义类别或提高相对信息权重来引导用户去关注某一类内容,这一点赋予了信息架构的设计管理者在设计制作完成后不断完善丰富交互内容的能力。

\subsection{信息架构组件}
信息架构的层次间可以互动,用户一般不常接触直接的层次,而是通过与信息架构组件间的交互完成信息互动。常见的组件有浏览帮手、搜索帮手、内容与任务,以及“不可见的”组件。

浏览帮手如网站的目录、导航栏、网站向导等,相当于文章的摘要和目录等,用以引导用户理解网站的大体结构。搜索帮手则是帮助带有特定目的浏览网站的用户更直接更快速地通过检索相关信息,直达自己所需信息的页面。内容和任务则是帮助带有探索目的的用户,以完成任务的形式去发现网站的功能,带有这种组件的网站一般为功能工具型网站而非展示类网站。“不可见的”组件即网站背后的算法和公式等,它们指导前台组件将用户导向更为合理高效的浏览途径上去。

在场景漫游中,内容与任务在场景内起主导作用,而在场景外则是导航与搜索起到引导用户踏足更多场景等作用。

\subsection{信息架构可视化}
信息架构不是藏在界面后面的东西,相反它出现在界面的各个角落。信息是一切操作的出发点和目的地,显然可见的信息架构有助于对此感兴趣的用户去进行深入体验。常见的信息架构的可视化体现在导航栏、搜索功能等上,但其他功能也会需要信息架构的支持,所以信息架构也被用来支持其他一切有功能的架构,以一种无形的形态在其他功能模块里实现自身的可视化。

\section{全景漫游的功能架构}
以功能目的为分类标准可讲全景漫游的功能分为:漫游功能、导航功能、搜索功能和记忆功能等。在实际应用中,功能的分块可因业务逻辑的复杂而增加或减少(如增添支付功能等)。以纯粹的体验角度而言,以上四项功能构成了使用者在全景漫游中绝大部分的行为模式,故在此将它们列举出来讨论是较为合适的。

\subsection{漫游功能}
全景漫游,重在漫游。虽是全景漫游的绝对重点,但其实是设计中最不需要考虑的一点。因为对于该功能的使用,人完全是通过日常经验作出的本能反应,设计只能迎合这种需求。例如,想要让人可以观察到一个景点的全部景致就需要获取到这个地方的全景图(如图\ref{fig:hongcun}),而不是简单地提供一张平面图片去让用户想象。而要做到良好的全景漫游体验,只是提供一张全景图片是远远不够的,需要加入众多的场景互动,例如上文的点击热点展开景点的相关信息等。这部分功能可以用另一种全景体验来形容,称为“增强现实”。所谓“虚拟现实”,就是在模拟出全景场景后,再添加“增强现实”的元素,以达到以假乱真甚至超过现实一般的体验。

\begin{figure}[htp]
\centering
\fbox{
\includegraphics[width=.7\textwidth]{hongcun}
}
\caption{全景图像示例}
\label{fig:hongcun}
\end{figure}

\subsection{导航功能}
在场景间切换需要导航功能,这种导航与常见网站与手机应用内的导航基本类似,但因全景漫游的场景是三维空间形式的,所以在场景内也通常会设置类似于一般游戏中的小地图,如图\ref{fig:minimap}。这种地图主要起到定位使用者在虚拟场景内位置的作用,原因是真实世界内使用者的位移和转向与游戏中并不完全对应,且因全景漫游体验的封闭性,人很难找到一个参考物,而地图这种形式可以方便地指明场景中角色于其他物体间的参照关系。当使用者转向时,小地图上的代表角色的图标也会随之转向,通过仔细观察并反向演算可以恢复自身的朝向感。

\begin{figure}[htp]
\centering
\fbox{
\includegraphics[width=.4\textwidth]{minimap}
}
\caption{一般游戏中的小地图}
\label{fig:minimap}
\end{figure}

\subsection{搜索功能}
搜索功能是非常常见的功能,在全景漫游中搜索功能与其他的基本类似,但因手部操作形式较为简单,故其输入文字信息的功能主要为语音识别来完成。同时,场景内的搜索功能可通过对物件的注视以获取悬浮提示的方式来选择搜索目标,即使用该物件作为搜索对象进行相关信息等检索。

\subsection{记忆功能}
记忆即储存相关信息,在探索行为中是非常重要的。记忆用户行为等目的不只是给予用户可查看的历史记录,更重要的是全景漫游中用户的操作目的性有时并不是那么明显,易造成误操作,通过还原场景功能可以撤销用户最近的若干次操作,同时可以保证场景漫游的流畅性。

系统记忆个人用户的身份信息也是有所必要的,因为每个人使用习惯与生理特性不同,系统应根据账号记忆不同账户使用者的使用习惯及设置,从使用过程中自适应用户的操作,以减少用户进行复杂设置的需求。


\section{全景漫游的功能模块}
\subsection{导航功能模块}
\subsection{场景特殊模块}
\subsection{多输入交互模块}
\subsection{热点模块}
