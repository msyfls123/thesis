\chapter{总结}

\section{主要工作与创新点}
本文在详细研究了全景漫游相关技术及市场现状的基础上,通过分析全景漫游与人的关系,针对现有全景漫游体验感和交互形式弱的问题,就以下内容进行了深入研究。

首先从全景漫游的技术特点入手,以人的视觉特征为对象分析并得出了全景漫游中视觉布局应偏下的特点。结合听觉中双耳效应的原理,论述了全景漫游中声音对视觉增强的辅助作用。全景漫游与人的肢体运动相结合分析,认识到全景漫游中漫游形式与感知设备的密切关系,并就全景漫游手部操控设备间进行了可用性方面的对比。阐述了全景漫游中“视觉辐辏”过程与适应性调节相冲突继而造成漫游中及漫游后晕眩的过程,提出减小物体间景深差距以缓解相关反应的设想。就全景漫游中意识的注意与持续集中方面进行论述,分析得出保持漫游中刺激强度和持续时间均衡以避免短时过强刺激影响使用的观点。通过双指缩放的案例说明记忆与思维在理解并记忆交互行为中的影响。

通过对可视化界面发展趋势的分析,表明了建立合理信息架构和交互模型的必要性。将全景漫游与真实世界的游乐场进行类比,阐述了熟悉的信息架构可以帮助用户更高效地交互的观点。从信息架构的三个组织方式:自顶向下、自底向上和不可见的组织形式出发,提出了信息架构组件化和可视化的观点,并借此建立全景漫游的基本功能模型:漫游、导航、搜索、记忆等。以部分特殊模块的设计为例,叙述了模块设计的规则特性,在多输入交互模块中提到了通过重力感应头部动作的形式可以进一步开发更多更适合全景漫游功能的交互形式。

针对全景漫游交互模型方面,以设备捕获的信息流为线索,举例说明了交互过程中等待反馈时间产生的原因与解决方法,并就信息修正的角度揭示了用户在交互过程中的主观能动性。通过对信息架构中三种组织形式的利用,构建了基于上下文感知、增强信息和多任务切换为基础全景漫游导航系统交互模型,提出了上下文语境中存在语境切换和小语境内部信息沟通的观点。通过阐述全景漫游中漫游路径形式的局限性,认识到场景切换中信息的表达性有待改善,提出了增强性场景切换标识的方式以实现场景中的直观化交互理念。

最后以某公司全景漫游系统开发为例,经过需求定义、功能架构定义和交互模型定义,通过定义常用术语帮助理解,并就导航界面、漫游界面和常用功能界面三个方面深入设计了全景漫游的应用原型。经过简单的程序开发,并经过分析采用了定性定量相结合的交互评价体系对用户主观评价和操作过程中的量化指标均提出了可行的评价形式,并以前文中设计的原型进行试评价,取得了一定的设计评价成果。

本文提出了以人为中心、可视化交互为主导的全景漫游交互设计思路,从人与全景漫游的关系至全景漫游中信息架构和交互模型的分析论证中分析了全景漫游交互设计中重点功能的实现与细节的处理,为全景漫游相关设计提供了相关分析思路和实践经验。

\section{有待改进的地方}
本文就全景漫游技术相关的可视化交互理论进行了详细的研究,提出了结合信息架构与交互模型的设计思路,并以实例开发为例进行实践和评价。基于本文基础可开展以下深入工作:
\begin{enumerate}
	\item 全景漫游中交互形式实现以现有技术为基础,但交互形似仍沿用计算机屏幕时代的交互特性,应提出更多适合可穿戴设备的多通道交互形式以增强交互的体验性。
	\item 本文提出的全景漫游交互模型应用场景为平台类的全景漫游应用,对单一应用的交互模型因应用场景限制不够深入,未来可针对特殊应用情形进行相应的扩展与补充。
    \item 本文实践部分采用的实现框架为 Web 方向,因作者对该方向技术较为熟悉,未来可借助本文中相关理论就其他平台全景漫游应用作相应衍生,希冀全景漫游技术得到更多平台的更多支持!
\end{enumerate}