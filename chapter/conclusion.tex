\chapter{总结}
本文从全景漫游发展的现状出发,列举并阐述了现有全景漫游技术的特点及需求,通过比较全景漫游多种应用的操作方式及交互流程发现其有待加强的地方。从人体生理与心理的相关特征特性出发,结合人机工程学和心理学观点深入剖析了全景漫游与人结合的难点和问题,力图以人为设计中心,加强人在全景漫游体验中的参与作用。从图形化人机界面到信息架构再到功能模块,以交互研究的方式结合相关实例说明交互模式在人机界面的重要作用。以实例分析的方式解释并总结了某一例全景漫游系统的交互设计及程序实现的过程,从需求分析到功能架构,以交互模型为指导原则进行可视化设计,完成了漫游系统的开发。在设计完成后,以收集用户体验数据的形式得到了宝贵的第一手资料,分析并论证了该交互设计中应用相关设计理论的可行性与成效。
