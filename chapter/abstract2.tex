% !TEX root = ../main.tex
\begin{cnabstract}
\begin{spacing}{1.0}
%\thispagestyle{empty}
\xiaosi
\setlength{\baselineskip}{20pt}
全景漫游是虚拟现实技术的核心,通过模拟真实场景的全景图或视频以及不同于传统平面媒介的多通道交互手段,用户可以通过全景漫游设备或是手机屏幕等体验到身临其境般的使用体验。将全景漫游相关技术应用于旅游、通讯、建筑、资源勘探等领域,可提升信息可视化程度,增强交互设计在计算机视觉及工业可视化方向上的应用价值。

本文从全景漫游发展的现状出发,列举并阐述了现有全景漫游技术的特点及需求,通过比较全景漫游多种应用的操作方式及交互流程发现其有待加强的地方。从人体生理与心理的相关特征特性出发,结合人机工程学和心理学观点深入剖析了全景漫游与人结合的难点和问题,力图以人为设计中心,加强人在全景漫游体验中的参与作用。

从图形化人机界面到信息架构再到功能模块,将全景漫游与真实世界中的游乐场场景进行类比,以交互研究的方式结合相关实例说明信息架构在人机界面的重要作用,论述了信息架构组件化和可视化的必要性。通过建立以漫游功能、导航功能、搜索功能和记忆功能为主的功能模型,列举了部分功能模块的实现要点。

在交互模型方面,以设备交互、导航交互和场景交互为切入点,以信息流的处理过程为例分析了交互操作形式的可反馈性的重要性,提出了用户在交互过程中需要进行信息修正的观点;详细阐述了信息架构的三种组织方式的应用场景,列举了自顶向下、自底向上和不可见的信息架构在模型中的具体体现:上下文感知用以建立导航间的上下文关系、增强信息用以减少用户变换注意的负担、多任务切换的合理形式以在用户记忆中留存相关信息利于语境的切换等。通过建立多层上下文语境增强语境间的联系,促进小语境间切换的适应性以及语境内部信息的流通。以场景中各种漫游路径的实现形式为基础,论述了场景中直观提供场景特殊标识存在的合理性与必要性,并给出了分层级特殊化的指示图标标识。

以实例分析的方式解释并总结了某一例全景漫游系统的交互设计及程序实现的过程,从需求分析到功能架构,以交互模型为指导原则进行可视化设计,主要从导航界面设计、漫游界面设计和功能界面设计三个方向举例交互原型设计,完成漫游系统的设计开发。同时,以 A-Frame 框架为基础,进行了简单的功能程序开发以供设计评价应用。在设计完成后,以收集用户体验数据的形式得到了用户交互操作行为的第一手资料,分析并论证了该交互设计中应用相关设计理论的可行性与成效。

本文提出了以人为中心、可视化交互为主导的全景漫游交互设计思路,从人与全景漫游的关系至全景漫游中信息架构和交互模型的分析论证中分析了全景漫游交互设计中重点功能的实现与细节的处理,为全景漫游相关设计提供了相关分析思路和实践经验。

\keywords{全景漫游、上下文感知、信息流、交互模型}
\end{spacing}
\end{cnabstract}


\begin{enabstract}
\begin{spacing}{1.0}
%\thispagestyle{empty}
\xiaosi
\setlength{\baselineskip}{20pt}
Panorama roaming is the basement of virtual reality technology. Users can immerse themselves in virtual environment by using panorama roaming device or mobile screen which is based on panorama image or video simulating real environment and multi-channel interaction methods excluding traditional interface media. Through reserach on panorama roaming, the implement value of interaction design in computer graphics and industrial visualization will be enhanced, as well as visualization of information in applyment fields including tourism, communication, architecture and resources exploration.

The thesis firstly lists and demonstrates characters and demands of its technology as current development of panorama roaming technology. The enhanceable details could be found by comparing serveral operative methods and interaction flow of ranorama roaming. The difficulties and problems between panorama roaming and human beings are explained according to ergonomics and psycology, with study on related human physical and psycologic characters. Interaction design will treate human as the middle part of roaming and enhance participation of human in roaming experience.

Methods of interaction researches connected to related applyment practice which are taken in flow of graphic interface, information architecture and functional module, comparison between panorama roaming and amusement park in orderto demonstrate fundemental function 
of information architecture in human-computer interface and its necessity of modularism and visualization of information architecture. The applyment details of specific functional modules are listed through making functional model based on roaming function, navigation function, search function and memory function.

In the part of interaction model, the importance of feedback in interactive operations is analyzed for examplex of information stream process at the point of device interaction, navigation interaction and environment interaction. The view point of user information revision is raised in interaction progress. The applyment fields of three types of interactive operation organization, as known as top-to-bottom, bottom-to-top and invisible forms, are demonstrated with their detailed implement: Context perception carries out context relationship between navigation layers, which enhances information to avoid rapid switching concern of user and keep memories in mind to benefit switch of context by making multi-task switch more reasonable. Multi-layer context is set up to enhance relationship between contexts which benefits both adaption of switch in minor context and context inner information transition. The reasonablity and necessity of existence of direct scene specific mark are demonstrated on base of implement of roaming routes, and layer-specialed guide icon marks are raised for example.

One case of panorama roaming system is detailed and concluded by practice analysis in interaction design and program implement. The work flows from requirement analysis to function architecture, visually developped in guide of interaction model. Interaction prototype design takes examples of navigation interface, roaming interface and function interface design contributing to development of roaming system. A-Frame which known as a web Javascript framework is applied in functional programme development with its extra components to meet the evaluation of design. User experience data is collected immediately after user interaction operation as the original profile which is used for analyzing feasibility and efficiency of related theories in interaction design.


The thesis raised one kind of human-oriented panorama roaming interaction design strategy leading by visualization interaction. Analysis on relationship between human and roaming, demostration of information architecture and interaction model draws the picture of panorama roaming interaction design function implement and detailed process. All above could be fundamental analysis strategy and practical experience for related designs.

\enkeywords{panorama roaming, context perception, information stream, interaction model}
\end{spacing}
\end{enabstract}
