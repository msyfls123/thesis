\chapter{全景漫游行业的市场现状}

\section{全景漫游技术总览}

全景漫游是由多种技术组合而成,其核心目的都是服务于将全景场景呈现给用户体验的过程。它是计算机仿真技术的重要方向之一,是计算机图形仿真、多媒体传感与网络通信等多种技术的综合,是一门富有挑战性的交叉技术前沿学科,具有广阔的研究领域。根据不同阶段所应用的技术类别,可以得出全景漫游的技术路径图,如图\ref{fig:process}所示,最左为真实场景,最右为用户,技术所起到的作用就是将场景与用户连接起来。

\begin{figure}[htp]
\centering
\fbox{
\includegraphics[width=.7\textwidth]{process}
}
\caption{全景漫游的技术路径}
\label{fig:process}
\end{figure}

全景漫游技术按载体分类可分为:
\begin{description}
\item[场景捕获硬件技术]视频捕获设备及其保障设备,用以采集全景漫游的视频、图片、音频等素材。
\item[场景呈现硬件技术] 主要包括 VR 眼镜、VR 互动手柄、VR 主机等。
\item[场景处理技术] 场景规划设计、场景素材的保存与传输以及相应程序开发等。
\item[场景还原与增强技术] 将场景素材与功能模块有机地整合,提供给用户身临其境般的全景漫游体验。
\end{description}

\subsection{场景呈现硬件技术}
相对而言,硬件技术相比于后两者进入门槛更高。目前市场上三大 VR 硬件厂商 OculusRift、HTCVive 和 PlayStationVR 几乎垄断了高端 VR 播放设备市场。之后加入 VR 市场的企业(如暴风、大朋、3Glasses、蚁视等)几乎都放低了姿态,推出了价格低廉但基本满足播放功能的入门级 VR 眼镜作为卖点。但随着 Google Cardboard 的推出,这款几乎没有成本可言的开源“硬件”迅速占领了低端 VR 硬件行业相当客观的市场份额。

\subsection{场景捕获硬件技术}
场景捕获技术的难点在于实时捕获,合成全景照片是一般手机都拥有的功能,大致原理就是捕获数张连续且相近的照片并通过算法进行合成球形场景。但实时场景捕获硬件的成本因其同步的特性更为高昂,目前市面上已有的厂商例如 Google JUMP、NOKIA OZO 等均是将现有平面镜头进行堆叠排布并通过后期处理合成虚拟场景,且这种方式因需要较多的镜头来堆砌同时刻的画面所以硬件成本非常高。目前市面上尚只有堆叠摄像头这一种方式进行全景场景捕获,而光场摄影等技术仍处在技术攻关阶段,相信不远的将来有可能出现消费级的场景捕获硬件。

\subsection{场景处理技术}
场景处理是目前全景漫游软件领域需要重点攻克的最大难关,但市面上各企业已基本做到自给自足的技术支持。全景漫游所依赖的基础是图像信号的捕获、加工与储存,而这些方面已有较多成功经验。目前 Web 端视频播放协议有以下两种:Real Time Messaging Protocol(实时消息传输协议,简称 RTMP)和 HTTP Live Streaming(HTTP 渐进下载,简称 HLS)。这两者是孑然不同的两种协议,而且 iPhone 等手机由于不自带支持 Flash 播放,一般考虑全端支持的流媒体播放会选用 HLS 作为传输协议,即对视频做切片,边播放边加载下一时段的切片。

\subsection{场景还原与增强技术}
场景还原与增强是与用户直接相连的部分,可以说前面的技术都是起为其保驾护航的功能。其中场景还原部分为计算机图形学的范畴,例如图像在空间上的曲率计算等,主要涉及到的技术有 OpenGL/WebGL 等,用于将二维图像还原成球状的场景。而场景增强技术则是本文所讨论的重点,其主要功能是为还原出的场景增添互动性,以使得用户可以自如地进行漫游体验。场景处理技术以软件形式作为载体,根据运行平台不同可分为直接编译代码至设备端的三维引擎软件和利用网页技术进行场景构建并可以运行于任意浏览器的 web 框架两种,以下以三款目前较为热门的软件框架进行介绍:

\paragraph{三维引擎软件:Unity 3D}

Unity 3D 是由 Unity Technologies 开发的一个使用脚本和即时编辑工具创建例如三维视频游戏场景、建筑可视化、实时三维动画等类型互动内容的多平台综合型游戏开发工具,是一个全面整合的专业游戏引擎。\endnote{百度百科.Unity 3D-百度百科. http://t.cn/RJDzA1F[EB]}

\paragraph{Web 全景漫游框架:Krpano}
Krpano 是一个小巧灵活的用来呈现各种全景图像和交互式的虚拟之旅的高性能全景查看器。可作为 Flash 和 HTML5 应用程序在 Web 上使用。并附有利用拖拽全景生成场景的 Krpano Tools 以供开发者快速生成用以展示的全景场景。本文将应用其进行部分设计案例的制作与演示。

\paragraph{Web 全景漫游框架:A-Frame}
A-Frame 是一个用 Web 技术构建 VR 体验的框架。使用 HTML 语言及实体组件来构建场景,可应用于 web/mobile 和其他多种设备端。本文将应用其进行部分设计案例的制作与演示。

\section{全景漫游市场前景}

2016 年,随着三大 VR 设备开始出售,未来 2-3 年 VR 设备普及率将快速提升。到达 2020 年,虚拟现实生态圈将初步形成,内容、服务等盈利模式逐步成熟,全球 VR 市场规模将达到 404 亿美元,VR 游戏市场规模将达到 149.5 亿美元。\endnote{VRZINC. 易观智库:2020年全球VR市场规模将达到404亿美元. http://vreyes.baijia.baidu.com/article/595977[EB]}

虚拟现实(VR/AR)产业市场具有良好前景,2015 年中 国虚拟现实行业市场规模为 15.4 亿元人民币。融资方面,国内 VR/AR 领域投资活跃度从 2015 年开始显著提升,2015 年第四季度和 2016 年第一季度的融资额均接近 10 亿元。其中显示设备融资占据首要地位,2015 年融资案例数量占比 30\%,融资额占比 69\%;内容制作的融资案例数量虽然占比 22\%,但融资额仅占比 6\%\endnote{赛迪智库.未来五年虚拟现实市场规模及前景展望. http://www.elecfans.com/vr/444138.html[EB]}。可见 VR 及全景漫游方向的内容制作生产领域仍处于起步阶段,与硬件厂商等差距较大,但同时其中也孕育了巨大的商机。

VR 内容开发受市场认可,线下体验馆增长迅速。由于中国 VR 市场主流设备仍以移动端 VR 眼镜为主,VR 视频内容的开发数量要远多于 VR 游戏内容。VR 平台上已有约 2700 款视频和 800 款游戏。预计 2020 年中国 VR 设备出货量 820 万台,用户量超过 2500 万人,见图\ref{fig:market}。

\begin{figure}[htp]
\centering
\fbox{
\includegraphics[width=.6\textwidth]{market}
}
\caption{2016-2020 年中国 VR 用户规模}
\label{fig:market}
\end{figure}

\section{全景漫游产业类别}
全景漫游产业按出发点可大体分为两类:
\begin{itemize}
	\item 以视频、游戏为主的虚拟内容提供商/平台商
	\item 涉及电商、教育、医疗、建筑等传统行业的行业融合应用服务商
\end{itemize}

短期而言,市场上比较活跃的是虚拟内容提供商,但长远而言,ToB 模式更利于产生更为健壮的全景漫游服务体系,对于高质内容的生产、分发和变现的模式也偏向于有传统大规模企业的行业服务商。

\section{全景漫游现有 APP 分析}
\subsection{Ascape}

Ascape 这款应用通过 360° 无缝全景视频让你通过景色的立体画面切身体验视频中呈现的著名景点,探索只有当地人才知道的隐藏宝藏的经历。并且镜头会伴随着你的移动,画面的位置和显示的角度都会随之改变,见图\ref{fig:ascape1}。

\begin{figure}[htp]
\centering
\fbox{
\includegraphics[width=.7\textwidth]{ascape2}
}
\caption{简洁的全景视频播放}
\label{fig:ascape1}
\end{figure}

同时,这款应用在 APP 传统界面上也继承了欧美一向以来的简约风格:“探索”页面采用卡片设计模式直观展示了新奇的场景;“发现”页面通过地图和名单两种模式展示了全球各地的场景选项;“我的旅行”也通过卡片模式展示了已下载的全景场景,见图\ref{fig:ascape2}。

\begin{figure}[htp]
\centering
\fbox{
\includegraphics[width=.7\textwidth]{ascape1}
}
\caption{简洁的全景视频播放}
\label{fig:ascape2}
\end{figure}

\subsection{Fulldive}

Fulldive 是一个智能手机与虚拟场景连接的平台,进入 APP 后即进入了一个全景漫游的世界。场景内包含常用的网络视频、本地视频/图片、VR 相机、VR 浏览器等应用,同时支持从其内部市场下载更多基于其开发的 VR 应用,见图\ref{fig:fulldive1}。

\begin{figure}[htp]
\centering
\fbox{
\includegraphics[width=.7\textwidth]{fulldive1}
}
\caption{Fulldive 内置的应用列表}
\label{fig:fulldive1}
\end{figure}

与 Ascape 不同之处在于 Fulldive 的操控完全是在虚拟全景中,所以在播放视频及操作选项时 Fulldive 有一些针对配戴 VR 眼镜者所特别提供的操作方式,见图\ref{fig:fulldive2}。例如,将视线聚焦在某个按钮上超过 3 秒后即视为按下了该按钮。这是一种与桌面鼠标和触屏点按都完全不同的交互形式,借鉴了鼠标的双击或是触屏的长按这两种操作,见图\ref{fig:fulldive3}。

\begin{figure}[htp]
\centering
\fbox{
\includegraphics[width=.6\textwidth]{fulldive2}
}
\caption{Fulldive 特有操作形式}
\label{fig:fulldive2}
\end{figure}

\begin{figure}[htp]
\centering
\fbox{
\includegraphics[width=.6\textwidth]{fulldive3}
}
\caption{Fulldive 视线停留模拟点击}
\label{fig:fulldive3}
\end{figure}

当然,这种模拟点击的操作只适用于”确认/取消“这种布尔判断的操作,用户需要输入大段的文字时则需有更高效的方式。Fulldive 结合了已日益成熟的语音识别技术,图\ref{fig:fulldive4}为实际操作 Fulldive 搜索功能并口述”中国地质大学“后的 APP 截屏。

\begin{figure}[htp]
\centering
\fbox{
\includegraphics[width=.5\textwidth]{fulldive4}
}
\caption{Fulldive 语音识别}
\label{fig:fulldive4}
\end{figure}

\subsection{VR X-Racer}

VR X-Racer 是一款简单的 VR 操控类飞行游戏,其操作方式为晃动设备(如 VR 眼镜或手机)来控制屏幕上的飞机躲避障碍物,见图\ref{fig:x-racer}。这是全景漫游技术在游戏上最直接的体现:几乎没有多余的操作,在游戏中飞机撞到障碍物坠毁后若干秒即自动重新开始游戏。但其最大的缺陷是无法长时间使用,不断晃动的全景屏幕容易使人产生眩晕、恶心等不良反应。
全景漫游与 3D 影片的体验是完全不同的,全景漫游将用户完全包裹在视频所构建的封闭环境中,而 3D 眼镜只是对屏幕这种有限区域进行了折射。两者的区别在于 3D 影片由于有周围环境作为铺垫不易使人完全沉浸其中,全景漫游则是在很短的时间内(20-30 秒)就使人沉浸在虚拟世界中,直至摘掉全景设备重新适应周围环境。

\begin{figure}[htp]
\centering
\fbox{
\includegraphics[width=.5\textwidth]{x-racer}
}
\caption{X-racer 游戏截图}
\label{fig:x-racer}
\end{figure}

\subsection{暴风魔镜 VR}

暴风影音是国内在全景漫游生态领域探索前沿的产品之一,其高端产品售价可达数千元,甚至与国外高端 VR 设备如 Oculus 和 GearVR 等价格相齐。与其硬件设备配套的则是一款叫做“暴风魔镜 VR”的手机 APP。

国内 APP 的发展方向一直是“大而全”,这款 APP 也不例外。暴风魔镜 VR 力图包括网络/本地视频、影音播放与 VR APP 等多种应用,形成自己的 VR 平台体系。同时支持普通的 APP 界面操作模式和全景漫游的 VR 模式。

在交互形式上与上文所列举的 Fulldive 类似,均为视线聚焦停留数秒视为确认,但缺少了语音识别输入大段文字的功能(在页面模式下支持手机输入法输入),如图\ref{fig:storm}。总体使用可满足基本的全景漫游体验,但识别速度较慢或误识别是比较严重的问题。

值得注意的一点是,暴风魔镜 VR APP 内场景下方有一个“归位”图标,触发后将会调整屏幕主视域至正对当前屏幕的位置,可类比于传统 APP 的“回到顶部”功能。这个功能很方便地起到了定位自身的作用,让用户不必盲目转动来寻找起始时正对着的界面。


\begin{figure}[htp]
\centering
\fbox{
  \includegraphics[width=.22\textwidth]{storm1}
}
\fbox{
  \includegraphics[width=.68\textwidth]{storm2}
}
\caption{暴风魔镜 VR 操作方式}
\label{fig:storm}
\end{figure}


\section{全景漫游行业现状总结}

全景漫游正处在行业发展的成长期,相关技术发展迅速,但同质化较为严重。众多厂商基于经济利益考量注重硬件设备的研发和更新换代,但软件质量与硬件质量仍有一定的差距,主要体现软件交互设计的质量上。

在全景漫游的交互设计上,国内外 APP 均有较多针对与不同与传统人机界面交互的创新点,例如前文列举的“视线停留数秒确定”、“语音识别”和“底部 Dock 栏”等。但交互可操作性相比于传统人机界面还有一定的差距,例如无法很快定位到所需内容、运动过程中不易操控及场景切换时语境丢失等问题。同时,在全景漫游的现有设计中,仍旧无法脱离面向传统人机界面的设计语境,在考虑保留用户习惯的同时难以创新出更为独特的交互模式,可以说是全景漫游交互设计亟待研究的难点。

全景漫游的行业发展离不开全景漫游的软硬件及生态环境的良好结合,只有如此才能留存有效用户并产生价值,故行业目前的重点就是在软硬件上均体现出应有的价值提升。其中,软件方面可在全景漫游领域的交互方向结合相关理论作出新的设计以改善用户体验。
