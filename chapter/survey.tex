\chapter{全景漫游行业的市场现状}
\section{全景漫游技术总览}

全景漫游技术按载体分类可分为:
\begin{enumerate}
\item{\emph{场景捕获硬件技术:}视频捕获设备及其保障设备,用以采集全景漫游的视频、图片、音频等素材。}
\item{\emph{场景呈现硬件技术:}主要包括 VR 眼镜、VR 互动手柄、VR 主机等。}
\item{\emph{场景处理技术:}场景规划设计、场景素材的保存与传输以及相应程序开发等。}
\item{\emph{场景还原与增强技术:}将场景素材与功能模块有机地整合,提供给用户身临其境般的全景漫游体验。}
\end{enumerate}

相对而言,硬件技术相比于后两者进入门槛更高。目前市场上三大 VR 硬件厂商 OculusRift、HTCVive 和 PlayStationVR 几乎垄断了高端 VR 播放设备市场。之后加入 VR 市场的企业(如暴风、大朋、3Glasses、蚁视等)几乎都放低了姿态,推出了价格低廉但基本满足播放功能的入门级 VR 眼镜作为卖点。但随着 Google Cardboard 的推出,这款几乎没有成本可言的开源“硬件”迅速占领了低端 VR 硬件行业相当客观的市场份额。

2016 年,随着三大 VR 设备开始出售,未来 2-3 年 VR 设备普及率将快速提升。到达 2020 年,虚拟现实生态圈将初步形成,内容、服务等盈利模式逐步成熟,全球 VR 市场规模将达到 404 亿美元,VR 游戏市场规模将达到 149.5 亿美元。\endnote[1]{http://vreyes.baijia.baidu.com/article/595977}