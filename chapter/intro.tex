\chapter{全景漫游技术与设计}
\section{全景漫游的目的}
古语云:眼观六路,耳听八方

这句话本意是形容人机制灵活,遇事能多方观察分析。而如今科技进步迅猛、社会发展气象万千,古时人们所向往的“千里眼”和“顺风耳”早已成为了现实(卫星雷达和手机电话),信息获取效率相比于古代有了巨大的提升。人们真正地做到了“眼观六路,耳听八方”。但这些技术往往只是片面地延展了人的感官能力,只是将人与物间某些距离拉近了,少有将人作为感官系统的中心来进行考量。在现代信息通讯过程中,人被排除在计算机的屏幕之外,人和真正的信息之间有一层薄薄的但难以直接触碰的玻璃阻隔。

古希腊的哲学家普罗泰戈拉说过:“人是万物的尺度,是存在的事物存在的尺度,也是不存在的事物不存在的尺度。“。借助现有科学技术,可以轻易地观摩到深入海底数公里的珊瑚礁分布、险象环生的热带雨林内生物的栖息环境、甚至是浩瀚无垠的太空中星象奇观。面对如此美景,只是观赏单一的图像帧序列来企图感受自然的奥妙总是觉得少了一份身临其境的体验。

人并不是一定要盯着一块显示屏。显然可得的想法就是利用显示屏将人的视线包裹起来,这样观察起来具有良好的主动性和体验感,同时规避了将人置于一些特殊环境所带来的附加成本或是风险因素。虽是虚拟,胜似现实,这就是“虚拟现实”的意义。而“全景漫游”则是“虚拟现实”中最重要的一部分。

\section{全景漫游的现状}
全景漫游是当今互联网行业发展最为迅速的技术之一,它是虚拟现实(即 VR 技术)的基础。沉浸性和交互性是其得到众多用户青睐的最主要原因,可供用户身临其境般体验的图形化界面,以及不同于传统平面媒介的交互手段共同促进了全景漫游技术的推广。
全景漫游解决了传统工业界对于空间概念呈现形式不足的痛点,通过定义场景的 360 度无缝全景图像,仿佛使人置身于制作者脑海中设计出的环境内,配合移动设备内置陀螺仪支持的重力感应功能,足不出户即可感受任何可以用图像表达的场景。80 天环游地球不再只是一个梦想,而且现今全景播放设备层出不穷,最便宜的 Google Cardboard 制作成本不足一元钱,人人 VR 的时代即将到来。

\section{全景漫游与设计结合的必要性}
设计行业向来紧跟时代发展潮流,全景漫游技术已经发展地比较成熟,因其本身具有互联网基因,故将其移植至设计产业的成本相较传统的工业技术会显著降低。传统产品原型技术将产品制作模型交付演示的时代可能即将被虚拟化技术所取代,相伴随的是设计成本的减轻以及设计迭代的加速。
将全景漫游技术引入设计行业的难点在于将全景展示的程式化模型转化为用户可探索、可理解、可学习的交互模型,现阶段的问题在于用户进入场景之后因场景与以往二维界面的割裂而导致心理产生的不适应性使其手足无措。值得研究的重点环节是降低用户的初次体验成本,加深用户记忆中对操作流程感悟的交互模型,最终达到用户无缝体验设计师所要传达的设计意图的目的。

\section{全景漫游中可视化交互理论研究的意义}
本课题旨在探索全景漫游场景中应用程序接口与用户心智模型的转化过程,通过分析场景程序执行的逻辑,用户操作的普适性规律以及人类学习行为的通常过程,将三者有机结合,以场景插件程序的形式进行封装,并提取出相应的 API 接口,以供本课题其后实例开发及其他对本课题感兴趣的开发者使用。
可视化技术对于工业发展的提升是显然的,从传统的制图学到近代的计算机辅助制图,再到现今的三维模拟技术,工业界向来拥抱可视化技术的前沿。可视化交互的意义就在于将图形化表述的产品按照其功能有机地组合起来,以可体验的形式呈现给用户/开发者。不再需要对产品的每部分组成有基础的了解也可认知其完成状态,也就意味着机械产品设计的成本进一步降低,普及到每个想要创造的人身上进而使其创造出更多产品的可能性变得现实。
